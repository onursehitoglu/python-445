\documentclass[trans,compress,xcolor=table]{beamer}
%\documentclass{article}
%\usepackage{beamerarticle}
%\usepackage{pstricks,pst-node,pst-uml} % PSTricks package
\usepackage[utf8]{inputenc}
\usepackage{listings}
\usepackage{multicol}
%\includeonlyframes{current}

\def\circtxt#1{$\mathalpha \bigcirc \mkern-13mu \mathtt #1$}
\def\colorline#1{\cr \noalign{\color{#1} \hrule height 1pt \vskip-3em}}
\def\colorfline#1{\noalign{\color{#1} \hrule height 1pt}}

\mode<article>
{
  \usepackage{fullpage}
  \usepackage{pgf}
  \usepackage{hyperref}
}

\mode<presentation>
{
  \usetheme{Antibes}
  \usecolortheme[rgb={0.9,0.4,0.1}]{structure}

  \setbeamercovered{transparent}
}


\title{Sofware Development with Scripting Languages:\\Python Crash Course}
\author{Onur Tolga Şehitoğlu}
\institute{Computer Engineering,METU}
\subject{Python Crash Course}
\date{24 February 2011}

\begin{document}
\lstset{language=Python,
        basicstyle=\scriptsize\ttfamily,
        keywordstyle=\color{blue!50!black}\bfseries,
        identifierstyle=\color{blue!60!green}\sffamily,
        stringstyle=\color{red!70!green}\ttfamily,
	commentstyle=\color{blue!30!white}\itshape,
        showstringspaces=true}

\setbeamercolor{pexample}{bg=orange!5!white,fg=black}%


 \frame{\maketitle}
\begin{frame}
\begin{multicols}{2}
\tableofcontents
\end{multicols}
\end{frame}

\section{Python Syntax}
\begin{frame}
\frametitle{Python Syntax}
\begin{itemize}
\item \alert{Indentation sensitive!!}
\item Code blocks are marked by indentation, not by explicit block markers (like \{ \} in C)
\item Each physical line indented with respect to previous one is assumed to be a new block
\item Blocks are only valid with block definitions (functions, loops, conditionals)
\item Backslash is used to join next line to current line \structure{continuation}
\item Bracket/paranthesis expressions explicitly introduce continuation
\end{itemize}
\end{frame}

\section{Values and Types}
\begin{frame}
\frametitle{Values and Types}
\begin{itemize}
\item \lstinline!type(..)! returns the type of any expression
\item Types
	\begin{itemize}
	\item Primitive: \lstinline!int! \lstinline!float! \lstinline!complex! \lstinline!bool! \lstinline!str! \lstinline!bytes!
	\item Composite: \lstinline!tuple!, \lstinline!list!, \lstinline!set!, \lstinline!dict!
	\item User defined and library classes
	\end{itemize}
\item Literals
	\begin{itemize}
		\item \lstinline!123! , \lstinline!231.23e-12! ,  \lstinline!2.3+3.534j!  , \lstinline!True!   , \lstinline!'hello'!  , \lstinline!"world"!  , \lstinline!b'ho\\xc5\\x9fgeldin'!
		\item \lstinline!0o432! (octal), \lstinline!0x43fe! (hex),
			\lstinline!0b0100110101! (binary)
		\item Composite: \lstinline!('ali',123)! \lstinline!['134',2,5]! \lstinline!\{'ali':4, 'veli':5\}! \lstinline!lambda x:x*x!
	\end{itemize}
\end{itemize}
\end{frame}

\section{Composite Types}
\begin{frame}
\frametitle{Composite Types}
\begin{itemize}
\item All types have their classes and class interface. Usually heteregeneous (each value may have a different data type)
\item \lstinline!help(classname)! is provided interactively
\item \structure{Tuples}: sequence of values separated by comma, enclosed in parenthesis. Immutable
\item \structure{Lists}: sequence of values, enclosed in brackets. Mutable
\item \structure{Sets}:  sequence of values separated by comma, enclosed in curly braces.
\item \structure{Dicts}: key-value pairs. key can be any hashable value (primitive)
\end{itemize}
\end{frame}

\defverbatim[colored]\codeoperators{
\begin{lstlisting}
+       -       *       **      /       //      %
<<      >>      &       |       ^       ~
<       >       <=      >=      ==      !=       is
[]      .       @
=       +=      -=      *=      /=      //=     %=
&=      |=      ^=      >>=     <<=     **=
or 		and 	not     del     in
\end{lstlisting}}
\section{Operators}
\begin{frame}
\frametitle{Operators and delimiters}
\begin{beamercolorbox}{pexample}
\codeoperators
\end{beamercolorbox}
\begin{itemize}
\item \lstinline!is! operands use same exact memory area
\item \lstinline!/! floating point division (flooring for integers in Python 2)
\item \lstinline!//! flooring division
\item \lstinline!**! power operator, right associative high precedence
\item \lstinline!del! delete an item from a data structure
\end{itemize}

\end{frame}


\section{Conditional}
\defverbatim[colored]\codecondition{
\begin{lstlisting}[escapeinside=\{\}]
if {\em condition\/} :
   {\em statements\/}
elif {\em condition\/} :
   {\em statements\/}
else:
   {\em statements\/}
\end{lstlisting}}

\begin{frame}
\frametitle{Conditionals}
\begin{beamercolorbox}{pexample}
\codecondition
\end{beamercolorbox}
\begin{itemize}
\item Indentation is required for blocks.
\item \lstinline!elif! and \lstinline!else! parts are optional
\item Conditional expression:\\
\lstinline!exp1 if condition else exp2!
\end{itemize}
\end{frame}

\section{While loop}

\defverbatim[colored]\codewhile{
\begin{lstlisting}[escapeinside=\{\}]
while {\em condition\/} :
   {\em loop body statements}
else:
   {\em termination statements}
\end{lstlisting}}
\begin{frame}
\frametitle{While loop}
\begin{beamercolorbox}{pexample}
\codewhile
\end{beamercolorbox}
\begin{itemize}
\item Loop body executed as long as condition is true
 \item \lstinline!else! part is optional
\item \lstinline!else! part is executed after loop terminates even if loop body is not executed at all.
\item When loop terminates without a condition test, i.e. by a \lstinline!break!,
	\lstinline!else! part is not executed.
\end{itemize}
\end{frame}
\section{For loop}

\defverbatim[colored]\codefor{
\begin{lstlisting}[escapeinside=\{\}]
for {\em var\/} in {\em iterable expression\/} :
   {\em loop body statements}
else:
   {\em termination statements}
\end{lstlisting}}
\begin{frame}
\frametitle{For loop}
\begin{beamercolorbox}{pexample}
\codefor
\end{beamercolorbox}
\begin{itemize}
\item Definite iteration over  a data structure
\item Iterable expression evaluated once. Then for each \lstinline!next()! value
     body of the loop is executed. The variable is assigned to value from next.
\item When all elements iterated, \lstinline!else! part is
 executed and loop terminates
 \item \lstinline!else! part is optional. Executed when loop terminates without \lstinline!break! or exception.
 \item lists, tuples, strings, dictionaries,... are iteratable objects
\end{itemize}
\end{frame}

\section{Escapes}
\begin{frame}
\frametitle{Escapes}
\begin{itemize}
\item \lstinline!break! terminates the last enclosing 
loop without executing \lstinline!else:! part if defined
\item \lstinline!continue! jumps to the beginning of next
   iteration (skips remaining part of the loop body)
\item \lstinline!try! statement is used to handle exceptions
\end{itemize}
\end{frame}


\section{Exceptions}
\defverbatim[colored]\codetry{
\begin{lstlisting}[escapeinside=\{\}]
try:
   {\em statements}
except {\em exceptionvalue\/} :
   {\em handler statements}
except {\em excetionvalue2\/} as {\em var\/}:
   {\em handler can refer to var for exception arguments}
except :
   {\em any exception handling}
\end{lstlisting}}
\begin{frame}
\frametitle{Exceptions}
\begin{beamercolorbox}{pexample}
\codetry
\end{beamercolorbox}

\begin{itemize}
\item exception values belong to \lstinline!exception! class
\item \lstinline!raise! statement can be used to raise an exception
\item If not handled exceptions stop execution
\item \lstinline!exception! class can be extended to define user-defined exceptions
\end{itemize}
\end{frame}

\section{Function definition}
\defverbatim[colored]\codefunction{
\begin{lstlisting}[escapeinside=\{\}]
def {\em functionname\/}({\em parameterlist\/}) :
   """ function documentation here
       continues....
   """
   {\em statements, function body\/}
   return {\em function return value\/}
\end{lstlisting}}
\begin{frame}
\frametitle{Function definition}
\begin{beamercolorbox}{pexample}
\codefunction
\end{beamercolorbox}

\begin{itemize}
\item Parameters can have default values as \lstinline!def f(x=0,y=0): ...!
\item When calling parameters can be explicitly chosen as \lstinline!f(y=2, x=4)!
\item Parameter passing is \structure{pass by reference}. The mutability of
      values are significant.
\item Assignment semantics is followed for parameter passing
\end{itemize}
\end{frame}

\section{Class definition}
\defverbatim[colored]\codeclass{
\begin{lstlisting}[escapeinside=\{\}]
class {\em classname\/}({\em optionalbaseclass\/}) :
   """ class documentation here
       continues....
   """
   cx = 0	# class member
   def __init__(self):
       """ this is constructor """
       self.x = 0    # how to create/access member variables
       self.y = 0
       classname.cx += 1	# class member update, 
   def increment(self):
       self.x += 1
   def _notprivate(self):
       pass          # no private members but methods starting with
                     # _ are private by convention

x = classname()     # how to create an instance
x.increment()       # call member
classname.increment(x) # other way of calling it
print(classname.cx)	# class members can be accessed as well
\end{lstlisting}}
\begin{frame}
\frametitle{Class definition}
\begin{beamercolorbox}{pexample}
\codeclass
\end{beamercolorbox}
\end{frame}
\section{Class Members}
\begin{frame}[fragile]
\frametitle{Class Members}
\begin{beamercolorbox}{pexample}
\begin{lstlisting}
class T:
    x      # class member
    def __init__(self):
      self.v = x   #!!!! Invalid
      self.v = self.x   # Valid as r-value, readonly access
      self.x = self.x + 1  # LHS instance member, LHS instance
      T.x += 1       # correct usage

a=T()
print(a.x)    # r-value, OK
a.x = 10      # l-value, creates an instance member
T.x = 10      # class member
\end{lstlisting}
\end{beamercolorbox}
\begin{itemize}
\item Class members work in class scope, not in object.
\item Instances can access them as r-value, not l-value.
\item Scope should be given explicitly, otherwise considered local variable.
\item \lstinline!classname.membername! is the correct way for l-value access.
\end{itemize}
\end{frame}
\begin{frame}[fragile]
\begin{itemize}
\item No private/public/protected in Python. All values are accessible.
\item Leading \lstinline!_! avoids importing from a module as \lstinline!from module import *!. Some frameworks libraries hide them with internal mechanisms.
\item Leading \lstinline!__! mangles name of method  as \lstinline!_ClassName__method! 
	when called outside. Other methods can call it with original name (with leading underscores). It is also used to avoid overriding due to inheritence.
\item \lstinline!@classmethod! decorator can be used to create methods getting class methods getting class as the parameter
\item \lstinline!@staticmethod! decorator can be used to create methods in class scope without any object/class parameter constraints.
\end{itemize}
\end{frame}
\begin{frame}
\begin{itemize}
\item \lstinline!self! is always the first parameter of the class method, it is
     passed as the firs parameter
\item \lstinline!__init__! is the constructor name
\item \lstinline!__str__! can be implemented to get string representing the object
\item \lstinline!__repr__! can be implemented to change how interpreter displays the object. 
	\lstinline!str()! calls \lstinline!repr()! when not implemented
\item \lstinline!__new__! is the class constructor (called before init)
\item \lstinline!isinstance(x, Myclass)! is instance check
\item \lstinline!issubclass(Cl1, Cl2)! is subclass check
\item \lstinline!super()! is the super class of the class
\item \lstinline!super().__init__()! calls the super class constructor. Not implicit.
\item \lstinline!superclassname.__init__(self, ...)! can be used as well.
\end{itemize}
\end{frame}

\section{Operator Overloading}
\begin{frame}
\frametitle{Operator Overloading}
\begin{itemize}
\item Operator overloading achieved through special
	member functions:\\
\lstinline'x * y' $\rightarrow$ \lstinline!x.__mult__(y)!\\
\lstinline'x / y' $\rightarrow$ \lstinline!x.__div__(y)!\\
\lstinline'x // y' $\rightarrow$ \lstinline!x.__floordiv(y)!\\
\lstinline'x > y' $\rightarrow$ \lstinline!x.__gt__(y)!\\
\lstinline'x[y]' $\rightarrow$ \lstinline!x.__getitem__(y)!\\
\lstinline'x[y]=z' $\rightarrow$ \lstinline!x.__setitem__(y,z)!\\
\lstinline'del x[y]' $\rightarrow$ \lstinline!x.__delitem__(y)!\\
\lstinline'x in y' $\rightarrow$ \lstinline!x.__contains__(y)!\\
\lstinline'x += y' $\rightarrow$ \lstinline!x.__iadd__(y)!\\
\lstinline'x.y' $\rightarrow$ \lstinline!x.__getattr__(y)!\\
\lstinline'x.y = z' $\rightarrow$ \lstinline!x.__setattr__(y,z)!\\
\lstinline'del x.y' $\rightarrow$ \lstinline!x.__delattr__(y)!\\
\end{itemize}

\end{frame}

\section{Variable Scope and Lifetime}
\begin{frame}
\frametitle{Variable Scope and Lifetime}
\begin{itemize}
\item Variables are local to enclosing block
\item Global variables have read only access unless
	they are used as l-value in the block
\item If variable used as an l-value a local variable
	is created and all read-only accesses preceding
	it gives an error
\item \lstinline'global' keyword is used to make a global variable
	available in a local block (read and update)
\end{itemize}
\end{frame}


\section{Assignment Semantics}
\begin{frame}
\frametitle{Assignment Semantics}
\begin{itemize}
\item Share semantics
\item Assignment copies reference, not data
\item Object assignment creates two variables denoting
	same object
\item Primitive values copied, objects shared (like Java)
\item Parameters pass by value for primitives, reference for
	objects
\item Constructors needed for copying \lstinline!list([1,2,3])!
\end{itemize}
\end{frame}

\section{Iterators}
\begin{frame}
\frametitle{Iterators}
\begin{itemize}
\item Iterators are used to iterate on data structures or create
	sequences
\item \lstinline!__iter__! method returns the iterator object
\item \lstinline!next! method gives the next object for the iterator
\item \lstinline!StopIteration! exception is rased on \lstinline!next! to
	end iteration
\end{itemize}
\end{frame}

\begin{frame}[fragile]
\begin{beamercolorbox}{pexample}
\begin{lstlisting}[escapeinside=\{\}]
for i in a:
   # loop body

{\rm\em is equivalent to:\/}

it = iter(a)
try:
   while True:
        i=next(it)
        # loop body
except StopIteration:
   pass
\end{lstlisting}\end{beamercolorbox}
\end{frame}

\defverbatim[colored]\codefibiter{
\begin{lstlisting}[escapeinside=\{\}]
class Fibonacci:
   def __init__(self,n):
	self.a = 0
	self.b = 1
	self.count = 0
	self.n = n
   def __iter__(self):
	return self
   def __next__(self):
	self.count += 1
	if self.count >= self.n:
		raise StopIteration
	self.a, self.b = self.b , self.a + self.b
	return self.a
\end{lstlisting}}
\begin{frame}
\begin{beamercolorbox}{pexample}
\codefibiter
\end{beamercolorbox}
\end{frame}

\section{Generators}
\begin{frame}
\begin{itemize}
\item In \structure{iterators} the state has to be kept in iterator object
\item Consider  a single instance of \lstinline!Fibonacci! iterated on a nested loop!
\item A correct implementation has to create a new instance for each \lstinline!iter()! call
\item Hard to write iterators on objects with \lstinline!next()! value is
	not trivial
\item \structure{Generators} automatically keep state of computation and continue where it left.
\item Use of \lstinline!yield! keyword is sufficient to write a generator
\item Each \lstinline!yield! corressponds to a \lstinline!next()!
\item Generator functions only \lstinline!return! without parameter to
	mark end of computation
\end{itemize}
\end{frame}

\begin{frame}[fragile]
\begin{beamercolorbox}{pexample}
\begin{lstlisting}[escapeinside=\{\}]
def fibonacci(n):
	a = 0
	b = 1
	count = 0
	while n>count:
		yield  b
		a,b = b,a+b
		count += 1
\end{lstlisting}\end{beamercolorbox}
Python creates all required intermediate objects and methods and return
a \lstinline!generator! object.
\end{frame}

\begin{frame}[fragile]
\frametitle{A Tree Example}
\begin{beamercolorbox}{pexample}
\begin{lstlisting}[basicstyle=\tiny\ttfamily,escapeinside=\{\}]
class BSTree:
	''' A binary search tree example'''
	def __init__(self):
		self.node = None
	def __setitem__(self, key, val):
		if self.node == None:	# empty tree
			self.node = (key, val)	# node content is a tuple
			self.left, self.right = BSTree(), BSTree()
		elif key < self.node[0]:	# not empty test key
			self.left[key] = val	# insert on left subtree
		elif key > self.node[0]:	
			self.right[key] = val	# insert on right subtree
		else: self.node = (key, val)	# update
	def __getitem__(self, key):
		if self.node == None:	# empty tree
			raise KeyError	# list, tuple also raise this
		elif key < self.node[0]:
			return self.left[key]
		elif key < self.node[0]:
			return self.right[key]
		else: return self.node[1]	# found, return the value
    def __str__(self):
        if self.node == None:   return "*"
        else:  return "[" + str(self.left)+str(self.node)+str(self.right) + "]"

a = BSTree()
for i in [6,2,8,3,9,1]:
	a[i] = i*i
print(a)
\end{lstlisting}
\end{beamercolorbox}
\end{frame}

\begin{frame}[fragile]
\frametitle{An  iterator on Tree Example}
\begin{beamercolorbox}{pexample}
\begin{lstlisting}[basicstyle=\tiny\ttfamily,escapeinside=\{\}]
# ... added to BST
    def _nextof(self, key):     # return min value >key
        if self.node == None:    return None
        elif key == None or key < self.node[0]:
            v = self.left._nextof(key)
            return self.node if v == None else v
        else:
            return self.right._nextof(key)

    def __iter__(self):
            return BSTree.BSTreeIter(self)  # new instance of nested class

    class BSTreeIter:
        def __init__(self, tree):
            self.tree = tree
            self.state = None
        def __next__(self):
            nextnode = self.tree._nextof(self.state)
            if nextnode == None:
                raise StopIteration
            else:
                self.state = nextnode[0]
                return nextnode

# main
a = BSTree()
#...  insert values etc.
for (k,v) in a:
    print(k,v)
\end{lstlisting}
\end{beamercolorbox}
\end{frame}

\defverbatim[colored]\codebstgen{
\begin{lstlisting}[escapeinside=\{\}]
# ... added to BSTtree
	def traverse(self):
		if self.node == None:
			raise StopIteration
		else:
			for (k,v) in self.left.traverse():
				yield (k,v)

			yield self.node

			for (k,v) in self.right.traverse():
				yield (k,v)

# main
a = BSTree()
#...  insert values etc.
for (k,v) in a.traverse():
	print(k,v)
\end{lstlisting}}
\begin{frame}
\frametitle{A  generator on Tree Example}
\begin{beamercolorbox}{pexample}
\codebstgen
\end{beamercolorbox}
\end{frame}


\section{String Processing}
\begin{frame}
\begin{itemize}
\frametitle{String Processing}
\item \lstinline!str()! class implements methods to process strings
\item \lstinline!string.split(delimeter)! is used to convert string to 
	an array of strings delimeted by delimeters.
\item \lstinline!string.join(array)! is used to convert a array of
strings to a concatanated string, seperated by the string object
\item \lstinline!map(function, sequence)! is used to apply function to
   all members of the sequence and return a list of return values
\item \lstinline!string.join(map(str, array))! will join string
	representation of all array types
\item \lstinline!+! concatanates to strings. All lexicographic
	comparisons are implemented as usual operators.
\item \lstinline!string.index(substr)! searches and returns position
	of substring in the string. \lstinline!find()!: same but
	returns -1 instead of exception.
\end{itemize}
\end{frame}


\section{Decorators}
\begin{frame}[fragile]
\frametitle{Decorators}
\begin{itemize}
\item Python decorators are evaluated during function/class definition
    and maps function/class definition into a new one
\item Decorators are functions returning a callable that replaces
    qactual function or class constructor.
\begin{lstlisting}[escapeinside=\{\}]
def dec(callable):
    def f():
      ..... use callable here
    return f

# ---
@dec
class cls(): {class defn here}
# equivalent to
class cls(): {class defn here}
  ...
cls = dec( cls )
# ---

obj = cls()     # actually calls f() inside dec()
\end{lstlisting}
\end{itemize}
\end{frame}

\begin{frame}[fragile]
\begin{itemize}
\item Decorators are used to alter functionalities of classes and functions for routine tasks.
\item A simple trace decorator:
\begin{lstlisting}
def trace(f,*p, **kw):
	def func(*p, **kw):
		print('entered', p, kw)
		r = f(*p, **kw)
		print('exitted', p, kw, r)
		return r
	return func

@trace
def f(x):
	return x+1

@trace
def g(a,b):
	if a < b: (a,b) = (b,a)
	return a if b == 0 else g(b, a % b)

print(f(3))
print(g(210,63))
\end{lstlisting}
\end{itemize}
\end{frame}

\begin{frame}[fragile]
\begin{itemize}
\item Example: an instance counting decorator:
\begin{lstlisting}
icount = {}
def counted(cls,*p,**kw):
    def func(*p, **kw):
        icount[cls] = icount[cls]+1 if cls in icount  else 1
        return cls(*p, **kw)
    return func

@counted
class A:
    def __init__(self,a):
        self.v = a
        
a = A(3)
\end{lstlisting}
\end{itemize}
\end{frame}

\begin{frame}[fragile]
\frametitle{Builtin Decorators}
\scriptsize
There are couple of useful decorators:
\begin{itemize}
\item \lstinline{@classmethod} makes a method expect a class argument as first argument. Method works in class scope instead of instance scope
\item \lstinline{@staticmethod} makes a method independent from object. It works as a normal function without object argument. It becomes a function in the class scope. Useful for auxiliary functions inside class definition.
\begin{lstlisting}[basicstyle=\tiny\tt]
class R:
    sumx, sumy = 0, 0		# class members: R.sumx,...
    def __init__(self, x,y):
        t = self._f(x,y)		# calls the staticmethod
        self.a, self.b = x/t, y/t
        R.sumx, R.sumy = R.sumx+x, R.sumy+y
    @classmethod
    def sums(cls):		# First argument is the class, not instance
        return cls.sumx, cls.sumy
    @staticmethod
    def _f(a,b):		# No extra argument 
        if a < b: a,b = b,a
        return a if b == 0 else R._f(b, a % b)  # recursive call

a = R(4,6)
print(R.sums(), a.sums())
print(a._f(72,48), R._f(72.48))
\end{lstlisting}

\end{itemize}
\end{frame}

\begin{frame}[fragile]
\small
\begin{itemize}
\item \lstinline!@property! decorator. Defines a member value that is calculated with a function.
\begin{lstlisting}[basicstyle=\tiny\tt]
class P:
    def __init__(self, name, surname):
        self.name, self.sname = name, surname
    @property
    def fullname(self):
        return ' '.join( (self.name , self.sname))

p = P('Bugs', 'Bunny')
print(p.fullname)
\end{lstlisting}
\item Also setter and deleter functions can be defined for setting the member and deleting it:
\begin{lstlisting}[basicstyle=\tiny\tt]
    #... continued ...
    @fullname.setter
    def fullname(self, value):   # gets the RHS as argument
        self.name, self.sname = value.split(' ')
	@fullname.deleter
	def fullname(self):			# removes the value
		self.name, self.sname = '', ''

p.fullname = 'Duffy Duck'
print(p.fullname)
del p.fullname
print(p.fullname)
\item Functional control over members can be implemented. (Invalid value assigment, precomputations etc.)
\end{lstlisting}

\end{itemize}
\end{frame}

\end{document}
